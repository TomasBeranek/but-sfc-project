\documentclass[10pt,xcolor=pdflatex,hyperref={unicode}]{beamer}
\usepackage{newcent}
\usepackage[utf8]{inputenc}
%\usepackage[czech]{babel}
%\usepackage[T1]{fontenc}
\usepackage{hyperref}
\usepackage{fancyvrb}
\usetheme{FIT}

\usepackage{setspace}
\usepackage{enumerate}
\usepackage[export]{adjustbox}
\usepackage{wrapfig}
\usepackage{multicol}
%%%%%%%%%%%%%%%%%%%%%%%%%%%%%%%%%%%%%%%%%%%%%%%%%%%%%%%%%%%%%%%%%%
\title{Vizualizace ACO \\pro hledání nejkratší cesty}

\author[]{
Tomáš Beránek\\
}

%\institute[]{Brno University of Technology, Faculty of Information Technology\\
%Bo\v{z}et\v{e}chova 1/2. 612 66 Brno - Kr\'alovo Pole\\
%login@fit.vutbr.cz}

\institute[]{xberan46@stud.fit.vutbr.cz\\
Fakulta informačních technologií Vysokého učení technického v Brně\\
%Bo\v{z}et\v{e}chova 1/2. 612 66 Brno - Kr\'alovo Pole\\
}

% České logo - Czech logo
% beamerouterthemeFIT.sty řádek 9: fitlogo1_cz

\date{13. prosince 2022}
%\date{\today}
%\date{} % bez data / without date

%%%%%%%%%%%%%%%%%%%%%%%%%%%%%%%%%%%%%%%%%%%%%%%%%%%%%%%%%%%%%%%%%%


\begin{document}


\frame[plain]{\titlepage}

\begin{frame}
\frametitle{Obecné informace}

Glabowski, M., Musznicki, B., Nowak, P. a Zwierzykowski, P. \emph{Shortest path problem solving based on ant colony optimization metaheuristic}. Image Processing & Communications. De Gruyter Poland. 2012, roč. 17, 1-2, s. 7.

\doublespacing
\begin{itemize}
    \item Řešení hostováno na Githubu: \url{https://github.com/TomasBeranek/but-sfc-project}
    \item Programovací jazyk: 
    \begin{itemize}
        \item Python3.8 -- pro Ubuntu,
        \item Python3.6 -- pro server Merlin (bez tooltipů).
    \end{itemize}
    \item GUI knihovna -- \emph{tkinter}.
\end{itemize}

\end{frame}

\begin{frame}
\frametitle{Algoritmus ShortestPathACO}
\doublespacing
\begin{itemize}    
    \item Větší podobnost reálným mravencům než u řešení \emph{TSP}.
    \singlespacing
    \item Mravenci vyráží z \emph{mraveniště} (počáteční uzel) a hledají \emph{jídlo} (koncový uzel), se kterým se vracejí zpět.
    \item Mravenci mají určitou \emph{rychlost} -- nepřeskakují mezi uzly.
    \doublespacing
    \item Výběr následujícího uzlu:
        \begin{multicols}{2}
        \noindent
        \begin{equation*}
        p_{ij} = \frac{q_{ij}}{\sum_{l:(i,l) \epsilon E}  q_{il}}
        \end{equation*}\notag
        \begin{equation*}
        q_{ij} = \tau^\alpha_{ij}\eta^\beta_{ij}
        \end{equation*}\notag
        \end{multicols}
        
        \begin{itemize}
            \item[] kde $\tau$ je množství feromonů a $\eta$ je váha (délka) hrany.
        \end{itemize}
\end{itemize}
\end{frame}

\begin{frame}
\frametitle{Algoritmus ShortestPathACO}
    \doublespacing
    \begin{itemize}    
        \item Mravenci se vracejí po stejné cestě \emph{bez smyček}.
        \item Feromony jsou vypouštěny \emph{postupně při návratu} mravence.
        \item \emph{Přírůstek feromonů} je počítán:
    \end{itemize}

    \begin{multicols}{4}
    \noindent
    \begin{equation*}
    \Delta \tau = 1
    \end{equation*}\notag
    \begin{equation*}
    \Delta \tau = \frac{1}{a_P}
    \end{equation*}\notag
    \begin{equation*}
    \Delta \tau = \frac{C}{a_P}
    \end{equation*}\notag
    \begin{equation*}
    \Delta \tau = \frac{a_{P_{best}}}{a_P}
    \end{equation*}\notag
    \end{multicols}

    \singlespacing
    \begin{itemize}
        \item \emph{Vypařování feromonů} probíhá každou sekundu a je nezávislé na rychlosti mravenců.
    \end{itemize}

    \doublespacing
    \begin{itemize}
        \item Stále se jedná o \alert{heuristiku}!
    \end{itemize}
\end{frame}

\begin{frame}
\frametitle{Nastavitelné parametry simulace}
\doublespacing
    \begin{itemize}
        \item Při spuštění programu:
        \begin{itemize}
            \item počet mravenců,
            \item podkladový graf (JSON formát).
        \end{itemize}
        \item Za běhu:
        \begin{itemize}
            \item typ výpočtu přírůstku feromonů,
            \item koeficient vypařování,
            \item rychlost mravenců,
            \item $\alpha$ -- vliv feromonů,
            \item $\beta$ -- vliv váhy (délky) hrany.
        \end{itemize}
    \end{itemize}
\end{frame}


\bluepage{Děkuji Vám za pozornost}

\end{document}
